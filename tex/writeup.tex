\documentclass{amsart}
\usepackage{hyperref, verbatim}
\begin{document}
	
	\large 
	\noindent Data Science Challenge Problem --- 2021 --- Boeing Global Services\\
	
	\noindent David Tersegno\\
	dtersegno@gmail.com\\
	\href{https://github.com/dtersegno}{Github}\\
	315 515 9983
	
	\subsection*{Data preparation} Numeric data was largely left as-is. Some that was not considered useful, like zip codes, or Vehicle Make (redundant with the models) was removed. Textual or list-like data describing car attributes was converted to over 200 dummy features.
	
	\subsection*{Predicting dealer list price} The data converted in the previous step was fed into a linear model. This model was then applied to the test set. The result was a fit with $R^2 \approx 0.8$.
	
	\subsection*{Predicting vehicle trim} The 29 listed trims were encoded as a 29-d vector. Data was duplicated to balance trim classes, and fed into a keras sequential model with a hidden layer. This model only achieved about 12\% accuracy with the balanced classes, but this is better than random choices of $1/29 \approx 0.03$. This is worse than simply assuming all models are Limited, which would give about 30\% accuracy assuming the frequency of Limited vehicles is about the same as in the data.
	\subsection*{source code} The sources are included here. However, they are in the form of ipython notebooks, so is inconvenient. \href{https://github.com/dtersegno/boeingds}{Please check my Github for the boeingds project repository to view more easily.}\\
	
	\begin{verbatim}
#!/usr/bin/env python
# coding: utf-8

# EDA
# ---
# 
# This notebook passes through each features in the used car data.
# 

# In[457]:


#import libraries
import os
import pandas as pd
import matplotlib.pyplot as plt
import seaborn as sns
import numpy as np

#display options
pd.options.display.max_columns = 40
get_ipython().run_line_magic('matplotlib', 'inline')
plt.style.use('dark_background')


# In[458]:


#import data
data_path = '../data/'
train_data_filename = 'Training_DataSet.csv'
test_data_filename = 'Test_Dataset.csv'

traindf = pd.read_csv(data_path + train_data_filename)
traindf.info()


# In[459]:


#look at top of the list
traindf.head()


# In[18]:


#look at basic statistics of numeric data
traindf.describe()


# ---
# # ListingID
# ---

# In[26]:


# 'ListingID' monotonically increases, approxmately linearly, with index.
testdf.ListingID.plot()


# In[21]:


#Increase of subsequent IDs is always positive but variable.
testdf.ListingID.diff().plot()


# In[24]:


#The distribution of the increase is exponentially decaying with larger skips.
testdf.ListingID.diff().hist(bins = 30, log = True)


# # SellerCity
# These all appear to be USA cities. Top represented cities largely not from the west coast?

# In[408]:


#plot number of instances of sales from the most represented cities
scvaluecounts = traindf.SellerCity.value_counts();
plot_limit = 20
plt.barh(scvaluecounts.index[:plot_limit], scvaluecounts[:plot_limit]);


# In[400]:


#how many cities are represented?
len(traindf.SellerCity.unique())


# In[409]:


#how many cities have more than one sale?
scvaluecounts[scvaluecounts > 1]


# In[410]:


#how many have more than 2 sales?
scvaluecounts[scvaluecounts > 2]


# In[424]:


#look at value counts of value counts.
scvcvc = scvaluecounts.value_counts()

#plot how many cities (y) have x sales
fig, ax = plt.subplots(figsize = (12,6))
ax.set_yscale('log')
plt.scatter(scvcvc.index, scvcvc)


# In[425]:


# plot how many sales come from a city with X sales (product of representation above with number of sales)
# the characteristic log shape for higher X comes from the values with only one city with that number.
fig, ax = plt.subplots(figsize = (12,6))
ax.set_yscale('log')
plt.scatter(scvcvc.index, scvcvc*scvcvc.index)


# In[431]:


#the above might be a bit deceptive. look at a histogram to account for the different density of numbers.

#number of cit
fig, ax = plt.subplots(figsize = (12,6))
plt.hist(scvcvc*scvcvc.index, bins = 50);


# In[443]:


# is there anything special about being sold from a high-sales city?
hi_sale_cutoff = 10
hi_sales_cities = traindf['SellerCity'].value_counts()[traindf['SellerCity'].value_counts() > hi_sale_cutoff].index


# In[444]:


hi_sales_cities


# In[448]:


#get sales statistics only from high-sales (>10 records) cities
traindf[traindf['SellerCity'].apply(lambda city: city in hi_sales_cities.to_list())].describe()


# In[447]:


#compare to statistics from low-sales cities.
traindf[traindf['SellerCity'].apply(lambda city: city not in hi_sales_cities.to_list())].describe()


# In[449]:


#with this (admittedly arbitrary) cutoff of cities with >10 sales in the set, cutting the records in about half, the mean dealer list price
# and std of sale price is about the same. 


# In[451]:


#look at mean dealer sales for each city. look at high sales cities only since the mean isn't too meaningful for lower sales count ones.
unique_cities = traindf['SellerCity'].unique()
city_means = []
for city in unique_cities:
city_means.append(
traindf.loc[traindf['SellerCity'] == city, 'Dealer_Listing_Price'].mean()
)
print(len(city_means), 'different city means')


# In[455]:


#plot city sale USD mean with number of sales
plt.scatter(traindf['SellerCity'].value_counts(), city_means)
plt.xlabel('Number of sale records in city')
plt.ylabel('city Dealer listing price mean')


# In[456]:


#it looks like number of sales and the sale mean aren't related. Although higher sale count cities appear to be more
#in the middle, this is likely an artifact of averaging more sales.


# ---
# # SellerIsPriv
# 
# If private seller. (Is a dealership a non-private seller?)

# In[62]:


#Only 14 of the listings are listed as private. This may cause overfitting.
priv


# In[70]:


# What are typical prices of these?
privateprices = traindf.loc[traindf['SellerIsPriv'] == 1, 'Dealer_Listing_Price' ]
privateprices.describe()


# In[94]:


privateprices.hist(bins = 8)
plt.axvline(privateprices.mean(), color = 'red')
plt.axvline(privateprices.describe()['50%'], color = 'green')


# In[95]:


#Compare this to the overall prices for the set later.


# # SellerListSrc

# In[98]:


#only a few different "seller listing source identifiers".
traindf.SellerListSrc.value_counts()


# In[99]:


#two listings are NaN
# These also have SellerZip, VehSellerNotes, and VehTransmission as NaN.
traindf.loc[traindf['SellerListSrc'].isna()]


# ---
# # SellerName

# In[104]:


#about 60% of the sellers only show once
len(traindf.SellerName.unique()), len(traindf.SellerName.unique())/len(traindf)


# In[130]:


#Look at number of sales of instances from each seller type.
sellernamecounts = traindf.SellerName.value_counts()[:40]
plt.figure(figsize = (12,12))
plt.barh(sellernamecounts.index, sellernamecounts, log = True)
plt.xticks(np.logspace(1,3,3));


# ---
# # SellerRating

# In[131]:


traindf.SellerRating.describe()


# In[139]:


#look at rating distribution
traindf.SellerRating.hist(bins = 20)


# In[143]:


#Seller rating with avg sales price isn't directly correlated
plt.scatter(
traindf.SellerRating,
traindf['Dealer_Listing_Price']
)
plt.xlabel('seller rating')
plt.ylabel("sale price")


# ---
# # SellerRevCnt

# In[150]:


#look at distribution of review counts. these include repeat-counted values from the same seller.
traindf.SellerRevCnt.hist(log = True)
plt.xlabel('review count')
plt.ylabel('number of sales with this count');


# In[155]:


#price with review count
plt.scatter(
traindf['SellerRevCnt'],
traindf['Dealer_Listing_Price']
)
plt.xlabel('Number of reviews')
plt.ylabel('sale price (USD)');


# ---
# # SellerState

# In[176]:


## this is redone with state names and regions following

#look at all unique seller states and rates
# statevcs = traindf['SellerState'].value_counts()
# plt.figure(figsize = (6,12))
# plt.barh(statevcs.index, statevcs, log = True)


# In[177]:


#import region table to look at regional representation
region_filepath = '../us-census-regions-divisions.csv'
regions = pd.read_csv(region_filepath)
regions.set_index('State Code', inplace=True)
#take a look to see it worked
regions.head()


# In[188]:


#join regions onto data to see representation of states and view
regiondata = traindf.join(regions, on = 'SellerState')[['SellerState', 'State', 'Region', 'Division']]
statesvcs = regiondata['State'].value_counts()
plt.figure(figsize=(6,12))
plt.barh(statesvcs.index, statesvcs, log = True)
plt.title("State representation (log)");
plt.grid(b = True, axis ='x')


# In[191]:


# look at region representations
regvcs = regiondata['Region'].value_counts()
plt.figure(figsize = (6,6))
plt.barh(regvcs.index, regvcs, log = False)
plt.title('Region representation')
plt.grid(b = True, axis = 'x')


# In[200]:


divvcs = regiondata['Division'].value_counts()
plt.figure(figsize = (6,6))
plt.barh(divvcs.index, divvcs, log = True)
plt.title('division representation (log)')
plt.grid(b = True, axis = 'x')
plt.xlim(1,3000);


# ---
# # SellerZip
# 
# Most of the listings do not have a zip --- this may be influenced by online sales.
# 
# Zip is highly related to seller and other location values.

# In[206]:


#look at common zips
traindf['SellerZip'].value_counts().head(20)


# In[202]:


traindf['SellerName'].value_counts()


# ---
# # VehBodystyle
# 
# Every listing is SUV body style. This is a useless column.

# In[211]:


traindf['VehBodystyle'].value_counts()


# ---
# # VehCertified

# In[218]:


certvcs = traindf['VehCertified'].value_counts()
plt.bar(['Not Certified', 'Certified'], certvcs)
plt.title("vehicle certified");


# ---
# # VehColorExt
# 
# There are a lot of exterior colors. It will be good to reduce these to subproperties like:
# 
# - general color
# - descriptive words like metallic, clearcoat, crystal, pearlcoat or Pearl Coat/Coat Pearl, tintcoat/tri-coat/tricoat/3-coat, tricoa,
# - texture words like Cashmere, Velvet, Frost, Pearl, ivory
# - extra sexy words like diamond, maximum, sangria (?), Stellar, Radiant, Passion, ** night edition **, midnight sky
#     - find out which extra descriptive words are true descriptors or simply marketing.
# 
# It also appears that roof camera information is included in some of these.

# In[237]:


def value_count_barplot(series, h = True, title = "plot title", log = False, figsize = (12,6)):
plt.figure(figsize = figsize)
this_value_counts = series.value_counts()
if h:
plot = plt.barh
else:
plot = plt.bar
plot(this_value_counts.index, this_value_counts, log = log)
plt.title(title)


# In[238]:


colorevcs = traindf['VehColorExt'].value_counts()
value_count_barplot(traindf['VehColorExt'], title = "exterior color representation", log = True, figsize = (6,40))


# ---
# # VehColorInt
# 
# Separate these into color and texture as well. Many are overlapping, for instance the repeat 'black' types.

# In[239]:


value_count_barplot(traindf['VehColorInt'], title = 'Interior color representation', log = True, figsize = (6,40))


# ---
# # VehDriveTrain
# 
# some of these are repeats, such as many different ways of writing All Wheel Drive.
# Some are overlapping or ambiguous, such as 2WD. Is that FWD or RWD?

# In[240]:


value_count_barplot(traindf['VehDriveTrain'], title = 'Drive Train', log = True)


# ---
# # VehEngine
# 
# Engine can be split into a few features:
# 
# - volume
# - number of cylinders
# - horsepower
# - special (HEMI, supercharged, MDS == Multi displacement, vvt variable valve timing, DI diesel direct, natural aspiration, mpfi multi port, DOHC dual overhead camshaft )
# 
# These are tough because most cars may have DOHC, certainly they all have "horsepower", but not every entry reports them.

# In[242]:


value_count_barplot(traindf['VehEngine'], title = 'Engine represenation', log = True, figsize = (6,20))


# In[ ]:


volumes = ['3.0L', '3.6L', '5.7L', '6.2L', '']


# ---
# # VehFeats
# 
# Lot of good info here. It appears to be well-written and organized --- information filled automatically rather than written manually like the previous. These should be flattened for value counts.

# In[465]:


#turn these objects into lists 
#
traindf['VehFeats'] = traindf['VehFeats'].dropna().apply(lambda entry: eval(entry))


# In[471]:


traindf['VehFeats'].dropna()


# In[466]:


traindf['VehFeats']


# In[467]:


# create a flat list of all feats
feat_list = [
feat for feats in traindf['VehFeats'].dropna() for feat in feats
]
#make that list a series to easily count values
featvcs = pd.Series(feat_list).value_counts()
#look at those value counts
featvcs.head()


# In[485]:


featvcs.head(80)


# In[468]:


traindf['VehFeats']


# ---
# # VehFuel
# 
# There is an unknown entry that is not used often, this can be mixed with NaN.

# In[297]:


traindf.loc[traindf['VehFuel'].isna(), 'VehFuel']


# In[298]:


traindf['VehFuel'].value_counts()


# ---
# # VehHistory
# 
# string split this over number of previous owners, as well as other features to track. Looks regulated (not manually entered)

# In[305]:


traindf['VehHistory'].head(10)


# ---
# # VehListdays

# In[318]:


#look at distribution of listed days
traindf['VehListdays'].hist(bins = 50, log = False)
plt.title('List days histogram ');
plt.xlabel('days listed')
plt.ylabel('count');
plt.show()

traindf['VehListdays'].hist(bins = 50, log = True)
plt.title('List days histogram (log)');
plt.xlabel('days listed')
plt.ylabel('count');
plt.show()


# In[322]:


#look at listed days with price. high price ones may go faster.
plt.scatter(traindf['VehListdays'], traindf['Dealer_Listing_Price'])
plt.xlabel('listed days')
plt.ylabel('listing price')


# ---
# # VehMake
# 
# Only Jeeps and Cadillacs. None missing. About twice as many Jeeps.

# In[326]:


makevcs = traindf['VehMake'].value_counts()
makevcs


# ---
# # VehMileage
# 
# What happens after 50,000 miles? Something special? There is a big dropoff in listings there.
# 
# The prices for the long-milage ones do not sink as low as the lowest pre-50k ones. These must be cars with something extra special.
# 

# In[332]:


traindf['VehMileage'].hist(bins = 30)
plt.title('Mileage histogram')
plt.xlabel("miles driven")
plt.ylabel('count')


# In[333]:


#mileage with price
plt.scatter(traindf['VehMileage'], traindf['Dealer_Listing_Price'])


# In[339]:


#cars with milage >50k are all cadillacs.
traindf.loc[traindf['VehMileage'] > 50000].head(40)


# ---
# # VehModel
# 
# All cars are either Jeep Grand Cherokee or Cadillac XT5s.

# In[340]:


traindf['VehModel'].value_counts()


# ---
# # VehPriceLabel
# 
# Many of these are missing. Missing ones may all be the same, probably a "bad deal".

# In[344]:


traindf['VehPriceLabel'].value_counts()


# In[361]:


traindf['Dealer_Listing_Price'].hist(density = True)
traindf[traindf['VehPriceLabel'].isna()]['Dealer_Listing_Price'].hist(density = True)


# In[351]:





# ---
# # VehSellerNotes
# 
# This will be tough to parse, as it's seller specific and manually entered. Much of this information is likely copied in other features (history, feats, colors,...).
# It may be valuable to look through some to see if there is any unusual words to grab onto.

# In[368]:


traindf['VehSellerNotes']


# In[367]:


traindf['VehSellerNotes'][4]


# ---
# # VehType
# 
# Useless. They're all used cars.

# In[370]:


traindf['VehType'].value_counts()


# ---
# # VehTransmission
# 
# It looks like most of these are like drivetrain --- manually entered, with a lot of overlap due to typos and other ways of writing. These can be reduced to a smaller number of transmissions.

# In[371]:


traindf['VehTransmission'].value_counts()


# ---
# # VehYear
# 
# Straightforward model year. This could be turned into dummies (each model is different). It is not the year it was manufactured, so it cannot be turned into a vehicle age. Mileage is perhaps a better estimator of effective age as it relates to value.

# In[372]:


traindf['VehYear'].value_counts()


# ---
# # Vehicle_Trim
# 
# there are a lot of vehicle trims. Some of these seem to overlap. For the sake of this problem, I will treat these as different classes, so the test predictions match the training data. However, in a "real situation", I'd investigate to see which of these are the same, and reduce the number of classes. 

# In[374]:


traindf['Vehicle_Trim'].value_counts()


# ---
# # Dealer_Listing_Price
# 
# the target. people don't like selling around \$30k?

# In[392]:


traindf['Dealer_Listing_Price'].hist(bins = np.linspace(15000,100000,86));
plt.title("dealer listing price histogram")
plt.xlabel("listing price (USD)")
plt.ylabel('count');


# ---

# In[398]:


correlation_features = ['SellerIsPriv', 'SellerRating', 'SellerRevCnt','VehCertified','VehListdays','VehMileage','VehYear','Dealer_Listing_Price']
corr = traindf[correlation_features].corr()
plt.figure(figsize = (12,12))
sns.heatmap(corr, annot=True)
\end{verbatim}

	\begin{verbatim}
#!/usr/bin/env python
# coding: utf-8

# Processing
# ---
# 
# This notebook prepares the data for modeling. Null values in numeric data are imputed. Lists of car features are cleaned and consolidated (for example, colors `black` and `Black` are considered the same). Strings and lists of string features are split, and the most common are made into dummies.

# In[1]:


#import libraries
import os
import pandas as pd
import matplotlib.pyplot as plt
import numpy as np
# import sklearn

#display options
pd.options.display.max_columns = 40
get_ipython().run_line_magic('matplotlib', 'inline')
plt.style.use('dark_background')


# In[2]:


#import data
data_path = '../data/'
train_data_filename = 'Training_DataSet.csv'
test_data_filename = 'Test_Dataset.csv'

traindf = pd.read_csv(data_path + train_data_filename)


#copy training data to a new dataframe to use for modeling
traindf_proc = traindf.copy()
traindf_proc.info()


# ---
# # Processing feature by feature

# # ListingID
# 
# drop it.

# In[3]:


#keep track to drop later.
columns_to_drop = ['ListingID']


# 
# 
# ## SellerCity
# Although SellerCity did not appear to influence the average dealer listing price, it is possible that patterns with SellerCity could be paired with other features to derive the value. This will not be effective in a linear regression model but could be grabbed onto in a decision tree / NN / etc.

# In[4]:


#make dummies for Seller Cities with the most sales in the training set.
#perhaps a bit arbitrary, but let's cut it off at cities above 30 sales.
#that's the first 20 common cities.
cities = traindf['SellerCity'].value_counts(ascending= False)[:20]
cities = cities.index
cities


# In[5]:


# takes a dataframe, a column to expand upon, and a list of values for that column to make dummies.
# any value in the column that is not in the dummy_list will be ignored

#the purpose of this is to make the same ordered columns in the training set as in any test set. Otherwise,
#the dummy columns may notinclude the same cities, if there is a different frequency distribution,
#or if a certain city isn't represented.
def make_specific_dummies(df, column, dummy_list):
#make a copy so we don't change the original
df2 = df.copy()
#remove any entries in column that aren't in the dummy list
for dummy_value in dummy_list:
df2[column + '_' + str(dummy_value)] = df2[column].apply(lambda entry: 1 if entry == dummy_value else 0)
return df2.drop(columns = column)



# In[6]:


traindf = make_specific_dummies(traindf, 'SellerCity', cities)
traindf.head()


# # SellerListSrc
# 
# Only two are nulls --- this will leave those as the only 0 for dummies. (no dropfirst)
# Again, `pd.get_dummies` could be a problem if there are a different order. Make dummy columns manually.

# In[7]:


sources = traindf['SellerListSrc'].dropna().unique()
sources


# In[8]:


traindf = make_specific_dummies(traindf, 'SellerListSrc', sources)


# ## SellerName
# Treat this like cities, with the most popular names marked. Again, arbitrarily for now, just picking the top 20 sellers.
# Some sellers might be tied with low prices for certain models.

# In[9]:


#not taking unique because there are so many. order by value count first, then take the first collection
sellers = traindf['SellerName'].value_counts(ascending = False)[:20]
sellers = sellers.index
sellers


# In[10]:


traindf = make_specific_dummies(traindf, 'SellerName', sellers)


# ## SellerState
# All 50 states are represented. The limit of US states means that we can drop one. Let's make it HI just because it has the fewest sales in this training set.

# In[11]:


states = traindf['SellerState'].value_counts(ascending = False).index[:-1]
states


# In[12]:


traindf = make_specific_dummies(traindf,'SellerState', states)


# # SellerZip, VehBodystyle
# drop.

# In[13]:


columns_to_drop.append('SellerZip')
columns_to_drop.append('VehBodystyle')


# # VehColorExt
# 
# colors to consolidate:
#    
#    - gray, platinum, silver, steel, granite, billet, billiet, Gy, sil, rhino
#    - black, Midnight Sky, Shadow, charcoal
#    - white, ivory
#    - blue
#    - brown, brownstone, mocha
#    - beige, beigh, cashmere, bronze, tan
#    - gold
#    - purple, amethyst, velvet
#    - deep red, dark red, deep cherry red, burgundy, sangria, deep auburn, maroon, 
#    - red, red horizon
#    - (black forest) green
#    - pink
#    
# textures / coats:
# 
#     - metallic, me
#     - pearl, pearlcoat, pearl-coat
#     - crystal
#     - clear, clearcoat
#     - 3-coat, tricoat, tri-coat
#     - tintcoat
#     
# nonvalues:
# 
#     - nan
#     - unspecified
#     - Not Specified

# In[14]:


colors = ['silver', 'black','white', 'blue', 'brown', 'tan', 'gold', 'purple', 'deep red', 'red', 'green', 'pink']

textures = ['metallic', 'pearl', 'crystal', 'diamond', 'clearcoat', 'tintcoat', 'tricoat']

silvers = ['gray', 'platinum', 'silver', 'steel', 'granite', 'billet', 'billiet', 'Gy', 'sil', 'rhino']
blacks = ['black', 'midnight','shadow','charcoal']
whites = ['white','ivory']
browns = ['brown','brownstone','mocha']
tans = ['beige','beigh','cashmere','brown','tan']
purples = ['purple','amethyst','velvet']
deepreds = ['deep red','dark red','deep cherry red','burgundy','sangria','deep auburn','maroon']


# In[15]:


#fill missing ext color values with blank string
traindf['VehColorExt'].fillna('', inplace = True)


# In[16]:


#create new dummy columns for basic colors. some will overlap (red, deep red, black forest green), but will largely be separated
color_synonyms = zip(colors, [silvers, blacks, whites, ['blue'], browns, tans, ['gold'], purples, deepreds, ['red'],['green'], ['pink']])
color_synonyms = list(color_synonyms)
for color in color_synonyms:
this_color = color[0]
these_synonyms = color[1]
traindf['ext_' + this_color] = 0
for synonym in these_synonyms:
traindf['ext_' + this_color] = traindf[['VehColorExt','ext_' + this_color]].apply(lambda entry: 1 if synonym in entry['VehColorExt'].lower() or entry['ext_' + this_color] else 0, axis = 1)


# In[17]:


#create dummy columns for textures
texture_synonyms = zip(textures,
[
['metal', 'me'],
['pearl'],
['crystal'],
['diamond'],
['clear'],
['tint'],
['3-coat', 'tricoat']
])
texture_synonyms = list(texture_synonyms)
for texture in texture_synonyms:
this_texture = texture[0]
these_synonyms = texture[1]
traindf['ext_' + this_texture] = 0
for synonym in these_synonyms:
#make the dummy value 1 if the synonym is in the description or if it's already 1
traindf['ext_' + this_texture] = traindf[['VehColorExt', 'ext_' + this_texture]].apply(lambda entry: 1 if synonym in entry['VehColorExt'].lower() or entry['ext_' + this_texture] else 0 , axis = 1)


# In[18]:


columns_to_drop.append('VehColorExt')


# # VehColorInt
# 
# Do the same for interior colors.
# 
# colors:
# 
#     - white
#     - beige, cream, cirrus
#     - black, carbon, graphite, ebony
#     - gray, pewter, aluminum, sterling
#     - tan
#     - ruby red
#     - sugar maple
#     - bronze
#     - blue, indigo, plum
#     - brown, sepia
#     - red
#     - jet
#     
#     
#     
# style:
# 
#     - sport
#     - accent/ accents
#     - mini-perf, mini-perfo
# 
# material:
# 
#     - leather
#     - suede
#     - titanium
#     - cloth
#     - sapelle, sapele
#     - aluminum

# In[19]:


traindf['VehColorInt'].fillna('', inplace = True)


# In[20]:


int_colors = ['beige',
'black',
'jet',
'gray',
'red',
'maple',
'blue',
'brown',
]

styles = ['sport',
'accent',
'perf']

materials = ['leather',
'suede',
'titanium',
'cloth',
'sapele',
'aluminum']


# In[21]:


#create dummy columns for interior colors 
int_color_synonyms = zip(int_colors,
[
['beige','cream','cirrus'],
['black','carbon','graphite','ebony'],
['jet'],
['gray','pewter','aluminum','sterling','steel'],
['red'],
['maple'],
['blue','indigo','plum'],
['brown','sepia']
])
int_color_synonyms = list(int_color_synonyms)
for color in int_color_synonyms:
this_color = color[0]
these_synonyms = color[1]
#create a column of zeros for this feature
traindf['int_' + this_color] = 0
#fill the new column if the scanned column has any of the synonyms
for synonym in these_synonyms:
#make the dummy value 1 if the synonym is in the description or if it's already 1
traindf['int_' + this_color] = traindf[['VehColorInt', 'int_' + this_color]].apply(lambda entry: 1 if synonym in entry['VehColorInt'].lower() or entry['int_' + this_color] else 0 , axis = 1)




# In[22]:




# scans through df[column] for anything in a synonym list, makes new columns with synonym headers.
# synonyms looks like 
# [ 
#   [ value1, [synonym1, synonym2...],
#   [ value2, [synonym1, synonym2...],
#   ...
# ]
def make_synonym_dummies(df, column, new_column_prefix, synonyms):
#not memory efficient, but safe.
df2 = df.copy()
for value in synonyms:
this_value = value[0]
these_synonyms = value[1]
#create a column of zeros for this feature
df2[new_column_prefix + '_' + this_value] = 0
#fill the new column if the scanned column has any of the synonyms
for synonym in these_synonyms:
#make the dummy value 1 if the synonym is in the description or if it's already 1
df2[new_column_prefix + '_' + this_value] = df2[[column, new_column_prefix + '_' + this_value]].apply(lambda entry: 1 if synonym in entry[column].lower() or entry[new_column_prefix + '_' + this_value] else 0 , axis = 1)
return df2

def make_inclusive_dummies(df, column, new_column_prefix, values):
df2 = df.copy()
for value in values:
df2[new_column_prefix + '_' + value] = df2[column].apply(lambda entry: value in entry.lower())    
return df2



# In[23]:


#style dummies
traindf = make_inclusive_dummies(traindf, 'VehColorInt', 'int', styles)


# In[24]:


#internal material dummies
traindf = make_inclusive_dummies(traindf, 'VehColorInt', 'int', materials)
traindf.columns[-9:]


# In[25]:


columns_to_drop.append('VehColorInt')


# ---
# # VehDriveTrain
# 
# Split these into two, 4x4/AWD and FWD/2WD. There are fine distinctions in cars in reality, but this data is mixed.
# 
# - AWD
#     - 4WD/AWD
#     - AllWheelDrive
#     - ALL WHEEL
#     - All-wheel drive
#     - AWD or 4x4
#     - 4x4/4WD
#     - 4x4
#     - All Wheel Drive
#     - ALL-WHEEL
#     - four wheel
#     - 4WD
# - 2WD
#     - front-wheel Drive
#     - 2WD
#     - FWD

# In[26]:


traindf['VehDriveTrain'].fillna('', inplace = True)


# In[27]:


drivetrains = ['awd', 'fwd']

awds = [
'4wd',
'awd',
'4x4',
'all-wheel',
'all wheel',
'four wheel',
'four-wheel'
]

fwds = [
'fwd',
'front-wheel',
'front wheel',
'2wd'
]

drivetrain_synonyms = zip(drivetrains,
[
awds,
fwds
])
drivetrain_synonyms = list(drivetrain_synonyms)


# In[28]:


drivetrain_synonyms


# In[29]:


#assign drivetrain dummies
traindf = make_synonym_dummies(traindf, 'VehDriveTrain', 'drivetrain', drivetrain_synonyms)


# In[30]:


columns_to_drop.append('VehDriveTrain')


# # VehEngine
# A lot of these engine values are included in `VehFeats`. Instead of regexing a bunch of engine values, I'll grab them from the feats.

# In[31]:


columns_to_drop.append('VehEngine')


# # VehFeats

# In[32]:


#turn these objects into lists 
#
traindf['VehFeats'] = traindf['VehFeats'].dropna().apply(lambda entry: eval(entry))


# In[33]:


# create a flat list of all feats
feat_list = [
feat for feats in traindf['VehFeats'].dropna() for feat in feats
]
#make that list a series to easily count values
featvcs = pd.Series(feat_list).value_counts(ascending = False)
#look at those value counts
featvcs.head()


# In[34]:


#grab the top feats to make a dummy list.
feats = featvcs.index[:80]


# In[35]:


traindf['VehFeats']


# In[36]:


for feat in feats:
traindf['feat_' + feat] = traindf['VehFeats'].dropna().apply(lambda featlist: feat in featlist)


# In[37]:


columns_to_drop.append('VehFeats')


# # VehFuel

# In[38]:


fuels = traindf['VehFuel'].unique()[:3]
fuels


# In[39]:


traindf = make_specific_dummies(traindf, 'VehFuel', fuels )


# # VehHistory

# In[40]:


traindf['VehHistory']


# In[41]:


#turn vehhistory into lists of strings
traindf['VehHistory'].fillna('', inplace = True)
traindf['VehHistory'] = traindf['VehHistory'].apply(lambda this_history: this_history.split(', '))


# In[42]:


# create a flat list of all hist
hist_list = [
hist for hists in traindf['VehHistory'].dropna() for hist in hists
]
#make that list a series to easily count values
histvcs = pd.Series(hist_list).value_counts(ascending = False)
#look at those value counts
histvcs


# In[43]:


#make hist dummies
hists = histvcs.index
for hist in hists:
traindf['hist_' + hist] = traindf['VehHistory'].dropna().apply(lambda histlist: hist in histlist)


# In[44]:


columns_to_drop.append('VehHistory')


# # VehListdays
# 
# Good to leave alone besides nas. Fill those with median.

# In[45]:


traindf['VehListdays'].fillna(traindf['VehListdays'].median(), inplace = True)


# # VehMake
# 
# only two makes.

# In[46]:


traindf = make_specific_dummies(traindf, 'VehMake', ['Jeep'])


# # VehMileage
# 
# numeric. Fine to leave alone. Two NaNs fill with median.

# In[47]:


traindf['VehMileage'].fillna(traindf['VehMileage'].median(), inplace = True)


# # VehModel
# 
# These are 1:1 with Make since there are only two models and two makes.

# In[48]:


columns_to_drop.append('VehModel')


# # VehPriceLabel
# 
# I'm guessing missing values are "bad". These will be the missing dummy.

# In[49]:


price_labels = traindf['VehPriceLabel'].unique()[:3]
price_labels


# In[50]:


traindf = make_specific_dummies(traindf,'VehPriceLabel', price_labels)


# # VehSellerNotes
# 
# I'd love to go through this to mark off important notes. But that will take time that I unfortunately don't have.

# In[51]:



columns_to_drop.append('VehSellerNotes')


# # VehType
# all the same.

# In[52]:


columns_to_drop.append('VehType')


# # VehTransmission
# 
# These are overwhelmingly 8-speed automatic. I don't think I will be able to extract much use out of this.

# In[53]:


columns_to_drop.append('VehTransmission')


# # VehYear
# 
# make dummies from these. A certain year may be favorable.

# In[54]:


years = traindf['VehYear'].unique()
years


# In[55]:


traindf = make_specific_dummies(traindf, 'VehYear', years)


# # VehTrim
# 
# the classification target.

# In[56]:


#save as a separate series
trim_data = traindf['Vehicle_Trim']
trim_data.to_csv('../data/train_trim_data.csv')

#drop
columns_to_drop.append('Vehicle_Trim')


# # finish up and save

# In[57]:


#drop columns
traindf.drop(columns = columns_to_drop, inplace = True)


# In[58]:


#turn all columns into floats for saving
traindf = traindf.astype(float)
traindf.info()


# In[59]:


#many rows are missing dealer list price, unfortunately. Drop these for now --- that's our target.
traindf = traindf.dropna()


# In[60]:


#save processed training data
processed_train_path = '../data/train_processed.csv'
traindf.to_csv(processed_train_path, index=None)


# # Repeat entire process for testing data.
# 
# Ideally a lot of this would be made with a single python script and some nice custom functions. For the sake of this challenge, I am copy-pasting.

# In[61]:


testdf = pd.read_csv(data_path + test_data_filename)
testdf.info()


# In[62]:


testdf = make_specific_dummies(testdf, 'SellerCity', cities)
testdf = make_specific_dummies(testdf, 'SellerListSrc', sources)
testdf = make_specific_dummies(testdf, 'SellerName', sellers)
testdf = make_specific_dummies(testdf,'SellerState', states)
testdf['VehColorExt'].fillna('', inplace = True)
for color in color_synonyms:
this_color = color[0]
these_synonyms = color[1]
testdf['ext_' + this_color] = 0
for synonym in these_synonyms:
testdf['ext_' + this_color] = testdf[['VehColorExt','ext_' + this_color]].apply(lambda entry: 1 if synonym in entry['VehColorExt'].lower() or entry['ext_' + this_color] else 0, axis = 1)
for texture in texture_synonyms:
this_texture = texture[0]
these_synonyms = texture[1]
testdf['ext_' + this_texture] = 0
for synonym in these_synonyms:
#make the dummy value 1 if the synonym is in the description or if it's already 1
testdf['ext_' + this_texture] = testdf[['VehColorExt', 'ext_' + this_texture]].apply(lambda entry: 1 if synonym in entry['VehColorExt'].lower() or entry['ext_' + this_texture] else 0 , axis = 1)
testdf['VehColorInt'].fillna('', inplace = True)
for color in int_color_synonyms:
this_color = color[0]
these_synonyms = color[1]
#create a column of zeros for this feature
testdf['int_' + this_color] = 0
#fill the new column if the scanned column has any of the synonyms
for synonym in these_synonyms:
#make the dummy value 1 if the synonym is in the description or if it's already 1
testdf['int_' + this_color] = testdf[['VehColorInt', 'int_' + this_color]].apply(lambda entry: 1 if synonym in entry['VehColorInt'].lower() or entry['int_' + this_color] else 0 , axis = 1)
testdf = make_inclusive_dummies(testdf, 'VehColorInt', 'int', styles)
testdf = make_inclusive_dummies(testdf, 'VehColorInt', 'int', materials)
testdf['VehDriveTrain'].fillna('', inplace = True)
testdf = make_synonym_dummies(testdf, 'VehDriveTrain', 'drivetrain', drivetrain_synonyms)
testdf['VehFeats'] = testdf['VehFeats'].dropna().apply(lambda entry: eval(entry))
for feat in feats:
testdf['feat_' + feat] = 0
testdf['feat_' + feat] = testdf['VehFeats'].dropna().apply(lambda featlist: feat in featlist)
testdf = make_specific_dummies(testdf, 'VehFuel', fuels )

testdf['VehHistory'].fillna('', inplace = True)
testdf['VehHistory'] = testdf['VehHistory'].apply(lambda this_history: this_history.split(', '))
for hist in hists:
testdf['hist_' + hist] = testdf['VehHistory'].dropna().apply(lambda histlist: hist in histlist)
testdf['VehListdays'].fillna(testdf['VehListdays'].median(), inplace = True)
testdf = make_specific_dummies(testdf, 'VehMake', ['Jeep'])
testdf['VehMileage'].fillna(testdf['VehMileage'].median(), inplace = True)
testdf = make_specific_dummies(testdf,'VehPriceLabel', price_labels)
testdf = make_specific_dummies(testdf, 'VehYear', years)


# In[63]:


testdf['VehFeats']


# In[64]:


# drop columns.
# it looks like vehicle trim isn't in the test data anyway.
columns_to_drop.remove('Vehicle_Trim')


# In[65]:



testdf.drop(columns = columns_to_drop, inplace = True)


# In[66]:


traindf.columns


# In[67]:


#The testing data has one fewer column --- that's the dealer price in training data!
testdf.columns


# In[68]:


# the feats introduced a lot of nulls --- fill these with 0
testdf.fillna(0., inplace = True)


# In[69]:


#turn bools into floats
testdf = testdf.astype(float)

#save the test data.
processed_test_path = '../data/test_processed.csv'
testdf.to_csv(processed_test_path, index=None)


# In[70]:


#from here, go to notebook 3 to train models.


\end{verbatim}
	\begin{verbatim}
#!/usr/bin/env python
# coding: utf-8

# model
# ---
# 
# This notebook creates a linear regression model from the training data and applies it to the test data.

# In[46]:


#import libraries
import os
import pandas as pd
import matplotlib.pyplot as plt
import numpy as np
from sklearn.linear_model import LinearRegression
from sklearn.model_selection import cross_val_score#, train_test_split


# In[47]:


train_data_path = '../data/train_processed.csv'
test_data_path = '../data/test_processed.csv'

traindf = pd.read_csv(train_data_path)
testdf = pd.read_csv(test_data_path)
traindf.info()


# In[61]:


testdf.info()


# In[48]:


#just to double-check, the training and testing data columns are the same.

traincols = traindf.columns.to_list()
traincols.remove('Dealer_Listing_Price')

traincols == testdf.columns.to_list()


# # predicting dealer listing price

# In[49]:


###### train-test split the training data to check
X = traindf.drop(columns = 'Dealer_Listing_Price')
y = traindf['Dealer_Listing_Price']

# actually, no need to tts. Just use cross_val_score. The linear fit is the same.
# Xtrain, Xtest, ytrain, ytest = train_test_split(X, y, random_state = 1916)


# In[50]:


#instantiate and fit a linear model to the training data using 5 cv folds

lr = LinearRegression()
cross_val_score(lr, X, y, cv = 5)


# In[51]:


# the fit is O.K. Use the linear regression on the test set and save.


# In[52]:


lr.fit(X, y)
preds = lr.predict(testdf)


# In[56]:


#the mean and std of the predicted values for the test set are similar to the dealer listing prices in the training set.
preds.mean(), preds.std(), y.mean(), y.std()


# In[58]:


# look at some predictions
testdf['Dealer_Listing_Price'] = preds
testdf['Dealer_Listing_Price']


# In[59]:


testdf


# # predicting trim

# In[98]:


#imports
from keras.layers import Dense
from keras.models import Sequential


# In[66]:


#load the trim data
trimdata = pd.read_csv('../data/train_trim_data.csv', index_col='Unnamed: 0')
trimdata.head()


# In[67]:


#slap the trim back on
traindf = traindf.join(trimdata)


# In[71]:


#remove nulls. A lot of the trims are missing.
traindf.dropna(inplace=True)


# In[78]:


# #get unique trims
# trims = list(traindf['Vehicle_Trim'].unique())


# In[133]:


# trims.index('Limited')


# In[135]:


# #turn trim strings into a vector and vice versa.

# def encode_trim(some_trim_string, unique_trim_list):
#     encoded_trim_vector = np.full(len(unique_trim_list), 0)
#     trim_index = list(unique_trim_list).index(some_trim_string)
#     encoded_trim_vector[trim_index] = 1
#     return encoded_trim_vector

# def decode_trim(trim_vector, unique_trim_list):
#     trim_index = list(trim_vector).index(1)
#     decoded_trim_string = unique_trim_list[trim_index]
#     return decoded_trim_string


# In[134]:


# #test them out
# test_decode = np.full(len(trims), 0)
# test_decode[5] = 1

# test_encode = decode_trim(test_decode, trims)
# test_encode


# In[194]:


#encode_trim(test_encode, trims)


# In[137]:


#turn all training trims into vectors

train_trims = pd.get_dummies(traindf['Vehicle_Trim'])


# In[150]:


trims = train_trims.columns


# In[208]:


traindf[traindf['Vehicle_Trim'] == 'Limited']


# In[211]:


get_ipython().run_cell_magic('time', '', "\n#duplicate data with underrepresented classes\nnewdf = pd.DataFrame([], columns = traindf.columns)\n\nfor vehicle_trim_class in traindf['Vehicle_Trim'].value_counts().index:\n    print(vehicle_trim_class)\n    keep_going = True\n    while keep_going:\n        newdf = pd.concat([newdf, traindf[traindf['Vehicle_Trim'] == vehicle_trim_class]])\n        if len(newdf[newdf['Vehicle_Trim'] == vehicle_trim_class]) > 600:\n            keep_going = False\n    ")


# In[213]:


len(newdf)


# In[250]:


num_inputs = len(newdf.columns)
num_outputs = len(trims)

#create a model
model = Sequential()
model.add(Dense(num_inputs, activation='relu'))
model.add(Dense(num_outputs*2, activation='relu'))
model.add(Dense(num_outputs, activation = 'softmax'))

model.compile(optimizer = 'adam', loss='categorical_crossentropy', metrics = ['accuracy'])


# In[251]:


#train it!

X = newdf.drop(columns = ['Dealer_Listing_Price','Vehicle_Trim'])
y = pd.DataFrame(newdf['Vehicle_Trim'].apply(lambda this_trim: encode_trim(this_trim,trims)).to_list())


# In[252]:


y


# In[255]:


get_ipython().run_cell_magic('time', '', 'epochs = 5120\nbatch_size = 256\nhistory = model.fit(\n    X,\n    y,\n    batch_size = batch_size,\n    epochs=epochs,\n    validation_split = 0.2,\n    verbose = False\n)')


# In[256]:


acc_hist, val_acc_hist = history.history['accuracy'], history.history['val_accuracy']

num_epochs_passed = len(acc_hist)
xes = np.linspace(0,num_epochs_passed,num_epochs_passed)
plt.plot(xes, acc_hist)
plt.plot(xes,val_acc_hist)


# In[257]:


#this is not good at all. Just train it on the unbalanced data.


# In[258]:


num_inputs = len(traindf.columns) - 2
num_outputs = len(trims)

#create a model
model = Sequential()
model.add(Dense(num_inputs, activation='relu'))
model.add(Dense(num_outputs*2, activation='relu'))
model.add(Dense(num_outputs, activation = 'softmax'))

model.compile(optimizer = 'adam', loss='categorical_crossentropy', metrics = ['accuracy'])


# In[259]:


#train it!

X = traindf.drop(columns = ['Dealer_Listing_Price','Vehicle_Trim'])
y = pd.DataFrame(traindf['Vehicle_Trim'].apply(lambda this_trim: encode_trim(this_trim,trims)).to_list())


# In[261]:


get_ipython().run_cell_magic('time', '', 'epochs = 512\nbatch_size = 256\nhistory = model.fit(\n    X,\n    y,\n    batch_size = batch_size,\n    epochs=epochs,\n    validation_split = 0.2,\n    verbose = False\n)')


# In[266]:


acc_hist, val_acc_hist = history.history['accuracy'], history.history['val_accuracy']

num_epochs_passed = len(acc_hist)
xes = np.linspace(0,num_epochs_passed,num_epochs_passed)
plt.plot(xes, acc_hist)
plt.plot(xes,val_acc_hist)


# In[272]:


model_preds = pd.Series([
trims[np.argmax(prediction)]
for prediction in model.predict(testdf.drop(columns = 'Dealer_Listing_Price'))
])


# In[273]:


model_preds.value_counts()


# In[280]:


predictions = pd.DataFrame(zip(preds, model_preds), columns = ['Dealer_List_Price', 'Vehicle_Trim'])
predictions.head()


# In[281]:


#save predictions
predictions.to_csv('../predictions.csv')


# In[ ]:





\end{verbatim}
\end{document}